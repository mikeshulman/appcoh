\documentclass[12pt]{amsart}

% PACKAGES ~~~~~~~~~~~~~~~~~~~

\usepackage{amsfonts, amsthm, amssymb, amsmath, stmaryrd, etoolbox, mathtools}
\usepackage[margin=1in]{geometry}
\usepackage{graphicx,caption,subcaption}
\usepackage{tikz}
\usetikzlibrary{matrix,arrows}

% NEW COMMANDS ~~~~~~~~~~~~~~~

% common math shorthands
\newcommand{\RR}{\mathbb{R}}
\renewcommand{\SS}{\mathbb{S}}
\newcommand{\ZZ}{\mathbb{Z}}
\newcommand{\zmodtwo}{\ZZ / 2 \ZZ}
\newcommand{\rptwo}{\RR \mathbf{P}^2}
\newcommand{\NN}{\mathbb{N}}
\newcommand{\QQ}{\mathbb{Q}}
\newcommand{\CC}{\mathbb{C}}
\newcommand{\from}{\colon}
%\newcommand{\to}{\rightarrow}
%\newcommand{\gets}{\leftarrow}
\newcommand{\xto}[1]{\xrightarrow{#1}}
\newcommand{\xgets}[1]{\xleftarrow{#1}}
\newcommand{\inv}{^{-1}}
\newcommand{\bydef}{\coloneqq}
\newcommand{\imp}{\Rightarrow}
\newcommand{\isequiv}{\Leftrightarrow}


% fonts
\newcommand{\edit}[1]{{\color{red} #1 }}
\newcommand{\cat}[1]{\mathbf{#1}}
\newcommand{\type}[1]{\mathtt{#1}}

% types
\newcommand{\tin}{\colon}
\newcommand{\A}{\type{A}}
\newcommand{\B}{\type{B}}
\newcommand{\C}{\type{C}}
\renewcommand{\P}{\type{P}}
\newcommand{\Q}{\type{Q}}
\newcommand{\BAC}{\B +_{\A} \C}
\newcommand{\Type}{\type{Type}}
\newcommand{\ap}{\type{ap}}
\newcommand{\inl}{\type{inl}}
\newcommand{\inr}{\type{inr}}
\newcommand{\glue}{\type{glue}}
\newcommand{\refl}{\type{refl}}
\newcommand{\code}{\type{code}}
\newcommand{\encode}{\type{encode}}
\newcommand{\decode}{\type{decode}}

% math operators
\DeclareMathOperator{\Hom}{Hom}
\DeclareMathOperator{\id}{id}
\DeclareMathOperator{\ob}{Ob}
\DeclareMathOperator{\arr}{arr}
\DeclareMathOperator{\im}{im}
\DeclareMathOperator{\Aut}{Aut}
\DeclareMathOperator{\Bij}{Bij}
\DeclareMathOperator{\Sub}{Sub}

% theorem styles
\newtheorem{lemma}{Lemma}
\newtheorem{thm}{Theorem}
\newtheorem{prop}{Proposition}
\newtheorem{cor}{Corollary}
\theoremstyle{remark}
\newtheorem{rmk}{Remark}
\theoremstyle{definition}
\newtheorem{defn}{Definition}
\newtheorem{ex}{Example}



%%%%%%%%%%%%%
% begin document
%%%%%%%%%%%%%

\begin{document}
	
\title{Proof Logic}
\maketitle

There are three related statements for the Borsuk-Ulam theorem

\begin{description}
\item[Statement $C$] $(f \from \SS^n \to \RR^n)$ $\imp$
  $(\exists x \in \SS^n. f(x) = f(-x))$. Symbolize this as $C_0
  \imp C_1$.
\item[Statement $O$] ($ g \from \SS^n \to \RR^n$ odd and
  cont. $\imp$ $\exists x \in \SS^n. g(x)=0$. Symbolize this as
  $O_0 \imp O_1$
\item[Statement $R$] $\exists h \from \SS^n \to \SS^{n-1}$ odd and
  cont. 
\end{description}

The Borsuk-Ulam theorem are the the three classically equivalent
statement \[C \isequiv O \isequiv - R.\] The logic underlying
classical strategy is to prove that equivalence as below, then prove
$-R$ and conclude $C$ which is the classic version of Borsuk-Ulam.

\begin{enumerate}
\item $C \models O$
\item $O \models C$
\item $O \models - R$ where we instead prove
  \begin{align*}
    R &\models -O &\\
    R &\models -(O_0 \imp O_1) &\\
    R &\models ( O_0 \wedge -O_1) &(R \models O_0 \text{ holds}) \\
    R &\models -O_1& (\text{prove this})
  \end{align*}
  Note these are two way equivalences in classical logic.
\item $-R \models O$
  \begin{align*}
    -R & \models O & \\
    -R & \models (O_0 \imp O_1) &\\
    -R & \models (-O_1 \imp -O_0) &\\
    -(-O_1 \imp -O_0) &\models R &\\
    (-O_1 \wedge O_0) &\models R & (\text{prove this})    
  \end{align*}
  Note these are two way equivalences in classical logic.
\end{enumerate}

Because the intuitionistic logic rule differ, we need to adjust our strategy to
accommodate for proving this in HoTT.



%%%%%%%%%%%%%
% end document
%%%%%%%%%%%%%

\end{document}
