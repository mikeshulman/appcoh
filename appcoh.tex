\documentclass{amsart}
% This is the preamble for the
% paper proving the Borsuk-Ulam
% theorem

% ~~~~~~~~~~~~~~~~~~~~~~~~~~~~~
% PACKAGES 
% ~~~~~~~~~~~~~~~~~~~~~~~~~~~~~
\usepackage{amsfonts}
\usepackage{amssymb}  
\usepackage{amsthm} 
\usepackage{amsmath} 
\usepackage{caption}
\usepackage[inline]{enumitem}
  \setlist{itemsep=0em, topsep=0em, parsep=0em}
  \setlist[enumerate]{label=(\alph*)}
\usepackage{etoolbox}
\usepackage{stmaryrd} 
\usepackage[dvipsnames]{xcolor}
  \definecolor{editcolour}{rgb}{0.7,0.1,0}
  \definecolor{hrefcolour}{rgb}{0,0,0.7}
  \definecolor{Amelia}{rgb}{1,0,0}
  \definecolor{Chandrika}{rgb}{0,1,0}
  \definecolor{Daniel}{rgb}{0,0,1}
\usepackage[]{hyperref}
  \hypersetup{%
    colorlinks,
    linkcolor={hrefcolour},
    citecolor={hrefcolour},
    urlcolor={hrefcolour}}
\usepackage{graphicx}
  \graphicspath{ {assets/} }
\usepackage{mathtools}
\usepackage{tikz, tikz-cd}
\usetikzlibrary{%
  matrix,
  arrows,
  shapes,
  decorations.markings,
  decorations.pathreplacing}

% ~~~~~~~~~~~~~~~~~~~~~~~~~~~~~~~
% NEW COMMANDS
% ~~~~~~~~~~~~~~~~~~~~~~~~~~~~~~~

% text formatting
\newcommand{\defn}[1]{\textbf{#1}} % to define a word
\newcommand{\cat}[1]{\mathrm{#1}} % category font
\newcommand{\type}[1]{\mathtt{#1}} % type font
\newcommand{\set}[1]{\mathbb{#1}} % mere set font

% sets and types
\newcommand{\A}{\type{A}}
\renewcommand{\AA}{\set{A}}
\newcommand{\B}{\type{B}}
\newcommand{\BB}{\set{B}}
\newcommand{\C}{\type{C}}
\newcommand{\CC}{\set{C}}
\newcommand{\D}{\type{D}}
\newcommand{\DD}{\set{D}}
\newcommand{\E}{\type{E}}
\newcommand{\EE}{\set{E}}
\newcommand{\F}{\type{F}}
\newcommand{\FF}{\set{F}}
\newcommand{\G}{\type{G}}
\newcommand{\GG}{\set{G}}
\renewcommand{\H}{\type{H}}
\newcommand{\HH}{\set{H}}
\newcommand{\I}{\type{I}}
\newcommand{\II}{\set{I}}
\newcommand{\J}{\type{J}}
\newcommand{\JJ}{\set{J}}
\newcommand{\K}{\type{K}}
\newcommand{\KK}{\set{K}}
\renewcommand{\L}{\type{L}}
\newcommand{\LL}{\set{L}}
\newcommand{\M}{\type{M}}
\newcommand{\MM}{\set{M}}
\newcommand{\N}{\type{N}}
\newcommand{\NN}{\set{N}}
\renewcommand{\O}{\type{O}}
\newcommand{\OO}{\set{O}}
\renewcommand{\P}{\type{P}}
\newcommand{\PP}{\set{P}}
\newcommand{\Q}{\type{Q}}
\newcommand{\QQ}{\set{Q}}
\newcommand{\R}{\type{R}}
\newcommand{\RR}{\set{R}}
\renewcommand{\S}{\type{S}}
\renewcommand{\SS}{\set{S}}
\newcommand{\T}{\type{T}}
\newcommand{\TT}{\set{T}}
\newcommand{\U}{\type{U}}
\newcommand{\UU}{\set{U}}
\newcommand{\V}{\type{V}}
\newcommand{\VV}{\set{V}}
\newcommand{\W}{\type{W}}
\newcommand{\WW}{\set{W}}
\newcommand{\X}{\type{X}}
\newcommand{\XX}{\set{X}}
\newcommand{\Y}{\type{Y}}
\newcommand{\YY}{\set{Y}}
\newcommand{\Z}{\type{Z}}
\newcommand{\ZZ}{\set{Z}}

% other sets and types
\newcommand{\ZZtwo}{\ZZ/2\ZZ} % Z mod two
\newcommand{\RP}{\type{RP}} % type real projective
\newcommand{\RRP}{\set{R}\textrm{P}} % set real projective
\newcommand{\KZtwo}[1]{K(\ZZtwo,#1)}

% symbols
\renewcommand{\equiv}{\simeq}
\newcommand{\bydef}{\coloneqq} % definitional equality
\newcommand{\tin}{\colon} % \in for types
\newcommand{\ltrunc}{\left| \left|}
\newcommand{\rtrunc}{\right| \right|}
\newcommand{\trunc}[1]{\ltrunc #1\rtrunc}
\newcommand{\shape}{\int}

% arrows
\newcommand{\from}{\colon}
\newcommand{\rel}{\nrightarrow}
\newcommand{\To}{\Rightarrow}
\newcommand{\xto}[1]{\xrightarrow{#1}}
\newcommand{\xgets}[1]{\xleftarrow{#1}}

% operators
\DeclareMathOperator{\Hom}{Hom}
\DeclareMathOperator{\id}{id}
\DeclareMathOperator{\colim}{colim}

% commenting
\newcommand{\Amelia}[1]{{\color{Amelia} AMELIA: #1}}
\newcommand{\Chandrika}[1]{{\color{Chandrika} CHANDRIKA: #1}}
\newcommand{\Daniel}[1]{{\color{Daniel} DANIEL: #1}}

% ~~~~~~~~~~~~~~~~~~~~~~~~~~~~~~~~~~~
% ENVIRONMENTS & COUNTERS
% ~~~~~~~~~~~~~~~~~~~~~~~~~~~~~~~~~~~
\newtheorem{theorem}{Theorem}[section]
\newtheorem{lemma}[theorem]{Lemma}
\newtheorem{proposition}[theorem]{Proposition}
\newtheorem{corollary}[theorem]{Corollary}

\theoremstyle{remark}
\newtheorem{remark}[theorem]{Remark}
\newtheorem{notation}[theorem]{Notation}

\theoremstyle{definition}
\newtheorem{example}[theorem]{Example} 
\newtheorem{definition}[theorem]{Definition}

\setcounter{tocdepth}{1} % Sets depth for table of contents. 


\begin{document}

% =================================================
% TITLE & AUTHORS
% =================================================
\title{%
        The Borsuk-Ulam theorm in real-cohesive homotopy
        type theory}   
\author{%
        Daniel Cicala \and Chandrika Sadanand \and Michael
        Shulman \and Amelia Tebbe}
\begin{abstract}
        Borsuk-Ulam!
\end{abstract}
\maketitle

% =================================================
% SECTION: COMMENTARY
% =================================================
\section*{Writing Notes}
\label{sec:writing_notes}

Writing assignments:
\begin{itemize}
\item Amelia---section 5
\item Chandrika---section 4
\item Daniel---sections 2 and 3
\end{itemize}

Formalizing the cohomology proofs will be determined later.


% =================================================
% SECTION: INTRODUCTION
% =================================================
\section{Introduction}
\label{sec:intro}

% =================================================
% SECTION: OVERVIEW OF REAL COHESIVE HOTT
% =================================================
\section{Overview of real-cohesive homotopy type theory}
\label{sec:rc-hott}

OUTLINE:
\begin{itemize}
\item
  HoTT as foundations
\item
  Intepreting AlgTop theorems in HoTT is obsructed by
  discontinuous functions
\item
  Relating continuous and discontinuous with flat and
  sharp, which are borrowed from cohesive topoi
\item
  Formalizing flat and sharp in HoTT + axioms needed,
  e.g. Rflat
\item
  Connecting sets used in AlgTop with HITs used in HoTT
  via shape
\end{itemize}

Homotopy type theory (HoTT) is an expression of a style of
mathematics that expands the notion of ``identity'' to
include logical identity, homotopy equivalence, and path
connectedness.  Experts call this style \emph{Univalence
  foundations}. And as foundations, there is an ambitious
program to encode all of mathematics in homotopy type
theory. There is a growning community working to realize
these ambitions and this paper belongs to this group.

Our present goal is to bring the classical theory of
algebraic topology into the fray, and in particular the
Borsuk-Ulam theorem.  However, the HoTT approach to
algebraic topology comes with on immediate challenge: the
presence of so many fixed point theorems where, in the
course of a proof, the fixed point must be specified
precisely, not only up to homotopy.  What is the problem
with this?  It is that homotopy type theory only works up to
homotopy. Compare, for instance, the topological circle
$$ \SS^1 \bydef \left\{ (x,y) \in \RR^2 | x^2 + y^2 = 1
\right\} $$ with the homotopy type theoretical circle
defined by a pair of constructors $ \type{base} $ and
$ \type{loop} \tin \type{base} = \type{base}$. One has
infinitely many points that can be described exactly and the
other has a single point. Brouwer's Fixed Point Theorem
illustrates this problem nicely. We provide its statement
and proof here for reference.

\begin{theorem}
  Let $ \DD^2 $ denote the topological disk $ \left\{ (x,y)
      \in \RR^2 | x^2 + y^2 \leq 1 \right\} $.  Any
  continuous map $ f \from \DD^2 \to \DD^2 $ has a fixed point.
\end{theorem}

\begin{proof}
  Suppose that $ f $ is continuous but does not have a fixed
  point, hence $ f(x) \neq x $ for all $ x \in \DD^2 $.
  For each $ x \in \DD^2 $, dray a ray from $ f(x) $ to $ x
  $. This ray intersects the circle in a point we denote by
  $ s(x) $. This defines a continuous function $ s \from
  \DD^2 \to \SS^1 $ with the property that $ s(x)=x $ for all $ x $ on
  the boundary of $ \DD^2 $.  That implies that the identity
  on $ \SS^1 $ factors as the inclusion $ \SS^1
  \hookrightarrow \DD^2 $ followed by $ s $.  Appying the
  fundamental group function $ \pi_1 $ to this factorization
  gives that the identity on $ \pi_1 \left( \SS^1 \right) =
  \ZZ $ factors through $ \pi_1 \left( \DD^2 \right) = 1 $
  which is absurd.
\end{proof}

Note how this proof relied on our precise specification of
the point $ s(x) $ on the circle.  This point cannot be
specified precisely in HoTT. Even if we did work with the
only homotopical point on the circle, that is with $ \S^1 $,
there is no way to relate $ \SS^1 $ to $ \S^1 $ inside of
type theory. Semantically speaking, this involves comparing
a topological space with $ \infty $-groupoids. This is done
using the fundamental $ \infty $-groupiod construction.  No
such construction exists in HoTT. This is the problem that
real-cohesive homotopy theory solves. It does so by
proposing to combine two already existing, but previously
unrelated, type semantics: topological and
$ \infty $-groupoidal. With this proposal, there are three
puzzles to be solved.

\begin{enumerate}
\item We need to define a model for a \emph{topological $ \infty
    $-groupoid}.
\item What rules or axioms can we equip HoTT with so that we
  can compare, for example, $ \SS^1 $ to $ \S^1 $.
\item Topology is incompatible with the law of the excluded
  middle, which is required to prove these classic fixed
  point theorems. How can we resolve this?
\end{enumerate}

Shulman's original paper on real-cohesive HoTT
\cite{shul:bfp} discusses the solution to these puzzles in
detail. Presently, we are content to simply say that the
Lawvere's theory of cohesion provides a solution. Of course,
we need to adapt cohesion to homotopy type theory and we
leave the description of this to Shulman, but we do provide
a high-level description of the role that cohesion plays.

A \emph{cohesive topos} is a pair of topoi equipped with a
string of adjunctions between them.  Lawever initially
called these topos






% =================================================
% SECTION: TRANLATING BORSUK ULAM TO HOTT
% =================================================
\section{Translating Borsuk-Ulam to homotopy type theory}
\label{sec:bu-to-hott}

OUTLINE:
\begin{itemize}
\item
  \textbf{Subsection 1.} Give statements for BU-classic,
  BU-odd, BU-retract) a la wikipedia. The proof strategy:
  show BU-retract implies BU-odd which is equivalent to
  BU-classic, then prove BU-retract. Give the proof for
  BU-retract.
\item
  \textbf{Subsection 2.} Translate the classical statement
  into propositions as types. We want to model classical proof.
  The failure of contrpositive rule in constructive
  logic---(not q implies not p) is (p implies not not
  q)---means our proof strategy is BU-retract implies not
  not BU-odd which is equivalent to not not BU-classic. But
  not not BU-classic is sharp BU-classic. Prove BU-retract. 
\item
  To close out the section, list the ingredients we need to
  prove BU-retract.
\end{itemize}


% =================================================
% SECTION: DEFINING RPn
% =================================================
\section{Topological and homotopical real projective spaces}
\label{sec:rpn}

OUTLINE:
\begin{itemize}
\item
  Define n-disks as both sets and types, the latter which is
  simply 1, since they're contractible. Show that $ \shape
  \DD = \D $
\item
  Define n-spheres as sets.  Use pushouts to glue
  disks together. Explain why we need to glue with a
  collar---i.e. the ``topology'' (as encoded by continuous
  paths $ \RR \to \X $ of a type $ \X $. Show, via Shulman,
  that $ \shape \SS^n = \S^n $
\item
  Define $ \RRP^n $ as sets using pushouts and collaring.
  Recall Bulcholtz and Egbert's definition of HIT $ \RP^n
  $. Prove that $ \shape \set{RP}^n = \RP^n $
\end{itemize}

\subsection{Defining $\set{RP}^n$}
We define $\set{RP}^n$ using push outs, tautological bundles, spheres, and a inductive process, following the work of Rijike [].

$\set{RP}^1$ is defined as the following push out, where $a(-1,x) = \frac{x}{4}$ and $a(1,x)= \frac{x}{4} + \frac{3}{4}$, and $b (\pm 1, x) := (0, a(\pm1, x))$.\\
\[\begin{tikzcd}
\set{S}^0 \times \set{I} \arrow[r, hook, "a"] \arrow[d, hook, "b" ] & \set{I} \arrow[d]\\
\{0 \} \times \set{I} \arrow[r] & \set{RP}^1
\end{tikzcd}
\]

Given $\set{RP}^{n-1}$, the tautological bundle

% =================================================
% SECTION: COHOMOLGY
% =================================================
\section{Cohomology}
\label{sec:cohom}

OUTLINE:
\begin{itemize}
\item 
  \textbf{Subsection 1.} 
  Define cohomology for $ \ZZtwo $ coefficients and the
  EM-spaces for $ \RRP^n $
\item
  \textbf{Subsection 2.} Show that we get a commutative
  graded ring structure for cohomology of any type $ \X $
  with $ \ZZtwo $-coefficients. Follow Brunerie's thesis.
\item
  \textbf{Subsection 3.} Compute $ \ZZtwo $-cohomology ring
  for $ \RRP^n $ using Mayer-Vietoris.  This needs us to
  first compute cohomology for disks and spheres.  
\end{itemize}


\subsection{Cohomology and EM-spaces for $\RRP^n$}


We follow a similar construction for cohomology as found in \cite{brunerie:thesis}, modifying their construction with $\ZZ$ coefficients to have coefficients in $\ZZtwo$. In order to define cohomology, we must first define Eilenberg-MacLane spaces $\KZtwo{n}$. Eilenberg-MacLane spaces $K(G,n)$ were defined for arbitrary group $G$ by Finster and Licata in \cite{fl:em}, and so we follow their construction. 

\begin{definition}
	For $n:\NN$, the type \defn{Eilenberg-MacLane space} $\KZtwo{n}$ is the $n$-truncated and $(n-1)$-connected pointed type defined by
	\[ \KZtwo{n}:=\left\{ \begin{array}{ll} \ZZtwo & \text{for }n=0\\
	\trunc{\Sigma^{n-1}\RRP^2}_n & \text{for } n\ge 1, 
	\end{array}\right.\]
	where $\Sigma^{n-1}$ indicates the reduced suspension. 
\end{definition}

\Amelia{G showed equivalence of $K_n$ and $\Omega K_{n+1}$ at this point. I don't think that's necessary given Finster and Licata.}

Given this construction of the EM spaces, we define cohomology in the following way. 

\begin{definition}
	For a type $\X$ and $n:N$, the \defn{$n$-th cohomology group of $\X$} is the type
	\[H^n(\X;\ZZtwo):= \trunc{\X\rightarrow \KZtwo{n}}_0.\] 
\end{definition}
\Amelia{B also defines reduced cohomology as well, not i'm not sure we need that.}


\subsection{Commutative Graded Ring Structure}

\subsection{Computing the Cohomology Ring of $\RRP^n$}




% =================================================
% SECTION: PROVING THE BORSUK ULAM THEORM
% =================================================
\section{The Borsuk-Ulam theorem}
\label{sec:borsuk-ulam}

OUTLINE:
\begin{itemize}
\item
  The proof is done by this point. Just put it all
  together and reconnect the dots for the reader.
\end{itemize}


% =================================================
% BIBLIOGRAPHY
% =================================================
\bibliographystyle{alpha}
\nocite{shul:bfp,brunerie:thesis,br:rp-hott}
\bibliography{assets/appcoh}


\end{document}