\documentclass{amsart}
% This is the preamble for the
% paper proving the Borsuk-Ulam
% theorem

% ~~~~~~~~~~~~~~~~~~~~~~~~~~~~~
% PACKAGES 
% ~~~~~~~~~~~~~~~~~~~~~~~~~~~~~
\usepackage{amsfonts}
\usepackage{amssymb}  
\usepackage{amsthm} 
\usepackage{amsmath} 
\usepackage{caption}
\usepackage[inline]{enumitem}
  \setlist{itemsep=0em, topsep=0em, parsep=0em}
  \setlist[enumerate]{label=(\alph*)}
\usepackage{etoolbox}
\usepackage{stmaryrd} 
\usepackage[dvipsnames]{xcolor}
  \definecolor{editcolour}{rgb}{0.7,0.1,0}
  \definecolor{hrefcolour}{rgb}{0,0,0.7}
  \definecolor{Amelia}{rgb}{1,0,0}
  \definecolor{Chandrika}{rgb}{0,1,0}
  \definecolor{Daniel}{rgb}{0,0,1}
\usepackage[]{hyperref}
  \hypersetup{%
    colorlinks,
    linkcolor={hrefcolour},
    citecolor={hrefcolour},
    urlcolor={hrefcolour}}
\usepackage{graphicx}
  \graphicspath{ {assets/} }
\usepackage{mathtools}
\usepackage{tikz, tikz-cd}
\usetikzlibrary{%
  matrix,
  arrows,
  shapes,
  decorations.markings,
  decorations.pathreplacing}

% ~~~~~~~~~~~~~~~~~~~~~~~~~~~~~~~
% NEW COMMANDS
% ~~~~~~~~~~~~~~~~~~~~~~~~~~~~~~~

% text formatting
\newcommand{\defn}[1]{\textbf{#1}} % to define a word
\newcommand{\cat}[1]{\mathrm{#1}} % category font
\newcommand{\type}[1]{\mathtt{#1}} % type font
\newcommand{\set}[1]{\mathbb{#1}} % mere set font

% sets and types
\newcommand{\A}{\type{A}}
\renewcommand{\AA}{\set{A}}
\newcommand{\B}{\type{B}}
\newcommand{\BB}{\set{B}}
\newcommand{\C}{\type{C}}
\newcommand{\CC}{\set{C}}
\newcommand{\D}{\type{D}}
\newcommand{\DD}{\set{D}}
\newcommand{\E}{\type{E}}
\newcommand{\EE}{\set{E}}
\newcommand{\F}{\type{F}}
\newcommand{\FF}{\set{F}}
\newcommand{\G}{\type{G}}
\newcommand{\GG}{\set{G}}
\renewcommand{\H}{\type{H}}
\newcommand{\HH}{\set{H}}
\newcommand{\I}{\type{I}}
\newcommand{\II}{\set{I}}
\newcommand{\J}{\type{J}}
\newcommand{\JJ}{\set{J}}
\newcommand{\K}{\type{K}}
\newcommand{\KK}{\set{K}}
\renewcommand{\L}{\type{L}}
\newcommand{\LL}{\set{L}}
\newcommand{\M}{\type{M}}
\newcommand{\MM}{\set{M}}
\newcommand{\N}{\type{N}}
\newcommand{\NN}{\set{N}}
\renewcommand{\O}{\type{O}}
\newcommand{\OO}{\set{O}}
\renewcommand{\P}{\type{P}}
\newcommand{\PP}{\set{P}}
\newcommand{\Q}{\type{Q}}
\newcommand{\QQ}{\set{Q}}
\newcommand{\R}{\type{R}}
\newcommand{\RR}{\set{R}}
\renewcommand{\S}{\type{S}}
\renewcommand{\SS}{\set{S}}
\newcommand{\T}{\type{T}}
\newcommand{\TT}{\set{T}}
\newcommand{\U}{\type{U}}
\newcommand{\UU}{\set{U}}
\newcommand{\V}{\type{V}}
\newcommand{\VV}{\set{V}}
\newcommand{\W}{\type{W}}
\newcommand{\WW}{\set{W}}
\newcommand{\X}{\type{X}}
\newcommand{\XX}{\set{X}}
\newcommand{\Y}{\type{Y}}
\newcommand{\YY}{\set{Y}}
\newcommand{\Z}{\type{Z}}
\newcommand{\ZZ}{\set{Z}}

% other sets and types
\newcommand{\ZZtwo}{\ZZ/2\ZZ} % Z mod two
\newcommand{\RP}{\type{RP}} % type real projective
\newcommand{\RRP}{\set{R}\textrm{P}} % set real projective
\newcommand{\KZtwo}[1]{K(\ZZtwo,#1)}

% symbols
\renewcommand{\equiv}{\simeq}
\newcommand{\bydef}{\coloneqq} % definitional equality
\newcommand{\tin}{\colon} % \in for types
\newcommand{\ltrunc}{\left| \left|}
\newcommand{\rtrunc}{\right| \right|}
\newcommand{\trunc}[1]{\ltrunc #1\rtrunc}
\newcommand{\shape}{\int}

% arrows
\newcommand{\from}{\colon}
\newcommand{\rel}{\nrightarrow}
\newcommand{\To}{\Rightarrow}
\newcommand{\xto}[1]{\xrightarrow{#1}}
\newcommand{\xgets}[1]{\xleftarrow{#1}}

% operators
\DeclareMathOperator{\Hom}{Hom}
\DeclareMathOperator{\id}{id}
\DeclareMathOperator{\colim}{colim}

% commenting
\newcommand{\Amelia}[1]{{\color{Amelia} AMELIA: #1}}
\newcommand{\Chandrika}[1]{{\color{Chandrika} CHANDRIKA: #1}}
\newcommand{\Daniel}[1]{{\color{Daniel} DANIEL: #1}}

% ~~~~~~~~~~~~~~~~~~~~~~~~~~~~~~~~~~~
% ENVIRONMENTS & COUNTERS
% ~~~~~~~~~~~~~~~~~~~~~~~~~~~~~~~~~~~
\newtheorem{theorem}{Theorem}[section]
\newtheorem{lemma}[theorem]{Lemma}
\newtheorem{proposition}[theorem]{Proposition}
\newtheorem{corollary}[theorem]{Corollary}

\theoremstyle{remark}
\newtheorem{remark}[theorem]{Remark}
\newtheorem{notation}[theorem]{Notation}

\theoremstyle{definition}
\newtheorem{example}[theorem]{Example} 
\newtheorem{definition}[theorem]{Definition}

\setcounter{tocdepth}{1} % Sets depth for table of contents. 


\begin{document}

% =================================================
% TITLE & AUTHORS
% =================================================
\title{%
        The Borsuk-Ulam theorm in real-cohesive homotopy
        type theory}   
\author{%
        Daniel Cicala \and Chandrika Sadanand \and Michael
        Shulman \and Amelia Tebbe}
\begin{abstract}
        Borsuk-Ulam!
\end{abstract}
\maketitle

% =================================================
% SECTION: COMMENTARY
% =================================================
\section*{Writing Notes}
\label{sec:writing_notes}

Writing assignments:
\begin{itemize}
\item Amelia---section 5
\item Chandrika---section 4
\item Daniel---sections 2 and 3
\end{itemize}

Formalizing the cohomology proofs will be determined later.


% =================================================
% SECTION: INTRODUCTION
% =================================================
\section{Introduction}
\label{sec:intro}

% =================================================
% SECTION: OVERVIEW OF REAL COHESIVE HOTT
% =================================================
\section{Overview of real-cohesive homotopy type theory}
\label{sec:rc-hott}

OUTLINE:
\begin{itemize}
\item
  HoTT as foundations
\item
  Interpreting AlgTop theorems in HoTT is obstructed by
  discontinuous functions
\item
  Relating continuous and discontinuous with flat and
  sharp, which are borrowed from cohesive topoi
\item
  Formalizing flat and sharp in HoTT + axioms needed,
  e.g. Rflat
\item
  Connecting sets used in AlgTop with HITs used in HoTT
  via shape
\end{itemize}

Homotopy type theory (HoTT) is an expression of a style of
mathematics that expands the notion of ``identity'' to
include logical identity, homotopy equivalence, and path
connectedness.  Experts call this style \emph{Univalence
  foundations}. And as foundations, there is an ambitious
program to encode all of mathematics in homotopy type
theory. There is a growing community working to realize
these ambitions and this paper belongs to this group.

Our present goal is to bring the classical theory of
algebraic topology into the fray, and in particular the
Borsuk-Ulam theorem.  However, the HoTT approach to
algebraic topology comes with an immediate challenge: the
presence of so many fixed point theorems where, in the
course of a proof, the fixed point must be specified
precisely, not only up to homotopy.  What is the problem
with this?  It is that homotopy type theory only works up to
homotopy. Compare, for instance, the topological circle
$$ \SS^1 \bydef \left\{ (x,y) \in \RR^2 | x^2 + y^2 = 1
\right\} $$ with the homotopy type theoretical circle
defined by a pair of constructors $ \type{base} $ and
$ \type{loop} \tin \type{base} = \type{base}$. One has
infinitely many points that can be described exactly and the
other has a single point. Brouwer's Fixed Point Theorem
illustrates this problem nicely. We provide its statement
and proof here for reference.

\begin{theorem}
  Let $ \DD^2 $ denote the topological disk $ \left\{ (x,y)
      \in \RR^2 | x^2 + y^2 \leq 1 \right\} $.  Any
  continuous map $ f \from \DD^2 \to \DD^2 $ has a fixed point.
\end{theorem}

\begin{proof}
  Suppose that $ f $ is continuous but does not have a fixed
  point, hence $ f(x) \neq x $ for all $ x \in \DD^2 $.
  For each $ x \in \DD^2 $, dray a ray from $ f(x) $ to $ x
  $. This ray intersects the circle in a point we denote by
  $ s(x) $. This defines a continuous function $ s \from
  \DD^2 \to \SS^1 $ with the property that $ s(x)=x $ for all $ x $ on
  the boundary of $ \DD^2 $.  That implies that the identity
  on $ \SS^1 $ factors as the inclusion $ \SS^1
  \hookrightarrow \DD^2 $ followed by $ s $.  Appying the
  fundamental group function $ \pi_1 $ to this factorization
  gives that the identity on $ \pi_1 \left( \SS^1 \right) =
  \ZZ $ factors through $ \pi_1 \left( \DD^2 \right) = 1 $
  which is absurd.
\end{proof}

Note how this proof relied on our precise specification of
the point $ s(x) $ on the circle.  This point cannot be
specified precisely in HoTT. Even if we did work with the
only homotopical point on the circle, that is with $ \S^1 $,
there is no way to relate $ \SS^1 $ to $ \S^1 $ inside of
type theory. Semantically speaking, this involves comparing
a topological space with $ \infty $-groupoids. This is done
using the fundamental $ \infty $-groupoid construction.  No
such construction exists in HoTT. This is the problem that
real-cohesive homotopy theory solves. It does so by
proposing to combine two already existing, but previously
unrelated, type semantics: topological and
$ \infty $-groupoidal. With this proposal, there are three
puzzles to be solved.

\begin{enumerate}
\item We need to define a model for a \emph{topological $ \infty
    $-groupoid}.
\item What rules or axioms can we equip HoTT with so that we
  can compare, for example, $ \SS^1 $ to $ \S^1 $.
\item Topology \Amelia{Type theory?} is incompatible with the law of the excluded
  middle, which is required to prove these classic fixed
  point theorems. How can we resolve this?
\end{enumerate}

Shulman's original paper on real-cohesive HoTT
\cite{shul:bfp} discusses the solution to these puzzles in
detail. Presently, we are content to simply say that the
Lawvere's theory of cohesion offers a solution. Of course,
we need to adapt cohesion to homotopy type theory and we
leave the description of this to Shulman, but we do provide
a high-level description of the role that cohesion plays.

A \emph{category of cohesive space} is a pair of categories
equipped with a string of adjunctions
\[
\begin{tikzcd}[row sep = 25pt]
  \cat{Spaces}
    \arrow[d, shift left = 1em, "f_\ast"']
    \arrow[d, shift right = 3em, "f_!"'] \\
  \cat{Sets}
    \arrow[u, shift left = 1em, "f^\ast"]
    \arrow[u, shift right = 3em, "f^!"] 
\end{tikzcd} 
\]
with $ f_! \dashv f^\ast \dashv f_\ast \dashv f^! $ and such
that $ f_! $ preserves finite products.  As Lawvere puts it,
the objects of $ \cat{Sets} $ should be thought of as
\emph{abstract sets} which
\begin{quote}
  ... may by conceived of as a bag of dots which are devoid
  of properties apart from mutual distinctness \cite{lawvere:cohesive}.
\end{quote}
On the other hand, the objects of $ \cat{Spaces} $ should be
thought of as abstract sets together with a sort of
\emph{cohesion} between the ``dots''. For our purposes, we
think of cohesion as a topology though, in reality, this
definition above axiomatizes the various forms that cohesion
may take, each functor playing a different role.  The
$ f_! $ functor tells us which points are ``stuck'' together
through the cohesion by returning a set of connected
components. The $ f^\ast $ functor endows a set with the
discrete topology. The $ f_\ast $ functor forgets the
topology of a space. The $ f^! $ functor endows a set with
the codiscrete topology on a set.  From this string of
adjoints, we get another adjoint string
$ \int \dashv \flat \dashv \sharp $ on $ \cat{Spaces} $
comprised of the \defn{shape operation}
$ \int \bydef f^\ast f_! $, the \defn{flat operation}
$ \flat \bydef f^\ast f_\ast $, and the \defn{sharp operation}
$ \sharp \bydef f^! f_\ast $.

To see how the axiomatic cohesion addresses the above
puzzles, we will speak in the language of sets and
categories instead of type theory. In other words, we
restrict out attention to the semantics of the relevant type
theory.

To solve the first puzzle, constructing a topological
$ \infty $-groupoid, we ask first that $ \cat{Spaces} $ and
$ \cat{Sets} $ are toposes. A cohesive topos is also a
\emph{local and locally connected topos} which can be
constructed using sheaves on a site that satisfy certain
properties. By expanding this construction to the
$ (\infty,1) $-category, we can obtain cohesive
$ (\infty,1) $-toposes using $ \infty $-sheaves on a site as
shown by Schreiber \cite{schreib:diff}.  The objects of a
resulting cohesive $ (\infty,1) $-topos are precisely the
topological groupoids we seek.

The second puzzle involves comparing a space with its
homotopy type.  Again modifying axiomatic cohesion, we
replace the categories $ \cat{Sets} $ and $ \cat{Spaces} $
with the $ (\infty,1) $-categories of $ \cat{Spaces} $ and $
\infty-\cat{Groupoids} $. We also replace the functors with
$ \infty $-functors.  The validity of this rests on work by
Schreiber \cite{schreib:diff}. In this setup, applying $
\shape $ to a space returns the fundamental $ \infty
$-groupoid, an excellent proxy for the homotopy type.  

Axiomatic cohesion also provides a solution to the final
puzzle: the failure of the \emph{continuous} excluded
middle. Given that we are working with topological objects,
we require that excluded middle holds continuously, but in
general it does not. Given a space $ X $ and subspace $ U $,
there is no continuous inverse to the inclusion
$ U + (X \setminus U) \to X $ because, even though the
underlying sets are the same, the topologies are
different. If we can introduce discontinuous functions
$ X \to U + (X \setminus U) $, then we can find a
discontinuous inverse to the inclusion, therefore obtaining
a modified, ``discontinuous'' law of the excluded
middle. Hence, the existence of a law of the excluded middle
in our context hinges on the introduction of such
discontinuous functions. To this end, recall that $ \flat $
retopolozes discretely and $ \sharp $ retopologizes
codiscretely. If $ \hom (X,Y) $ is the space of continuous
functions from $ X $ to $ Y $, then both $ \hom (\flat X,Y)
$ and $ \hom (X,\sharp Y) $ contain the discontinuous, by
which we mean not necessarily continuous, functions from $ X
$ to $ Y $.  

Moving towards syntax means introducing into homotopy type
theory the constructors that mirror the semantics of $ \int $,
$ \flat $, and $ \sharp $. Upon adding these constructors,
we obtain \emph{cohesive homotopy type theory}. The ``real''
part of name comes from an additional axiom included so that
we can capture the topology syntactically using continuous
paths from the reals. This axiom states
\begin{center}
\begin{quotation}
  A crisp type $ \type{A} $ is discrete if and only if the
  function that returns a constant path
  $ \type{A} \to (\RR \to \type{A}) $ is an equivalence.
\end{quotation}
\end{center}
Calling $ \type{A} $ a crisp variable means that we
perform constructions on it without regarding the topology,
such as defining maps $ \flat\type{A} \to \type{Y} $ or $
\type{X}\to\sharp\type{A} $.  When including this axiom
along with the syntactic versions of shape, flat, and sharp,
we get \emph{real-cohesive homotopy type theory}.















% =================================================
% SECTION: TRANLATING BORSUK ULAM TO HOTT
% =================================================
\section{Translating Borsuk-Ulam to homotopy type theory}
\label{sec:bu-to-hott}

OUTLINE:
\begin{itemize}
\item
  \textbf{Subsection 1.} Give statements for BU-classic,
  BU-odd, BU-retract) a la wikipedia. The proof strategy:
  show BU-retract implies BU-odd which is equivalent to
  BU-classic, then prove BU-retract. Give the proof for
  BU-retract.
\item
  \textbf{Subsection 2.} Translate the classical statement
  into propositions as types. We want to model classical proof.
  The failure of contrapositive rule in constructive
  logic---(not q implies not p) is (p implies not not
  q)---means our proof strategy is BU-retract implies not
  not BU-odd which is equivalent to not not BU-classic. But
  not not BU-classic is sharp BU-classic. Prove BU-retract. 
\item
  To close out the section, list the ingredients we need to
  prove BU-retract.
\end{itemize}

\subsection{The classical Borsuk-Ulam Theorem}

Outside homotopy type theory, the Borsuk-Ulam Theorem states that 

\begin{statement}\label{BUClassic}
If $f: \SS^n \rightarrow \RR^n$ is a continuous function, then there exists a point $x \in \SS^n$ such that $f(x) = f(-x)$, where $-x$ is the antipodal point of $x$. 
\end{statement}

Instead of proving this statement directly in homotopy type theory, we make use of two equivalent statements involving odd functions.

\begin{statement}\label{BUOdd}
Any odd continuous function $g: \SS^n \rightarrow \RR^n$ must have a zero.
\end{statement}

To see that Statement \ref{BUOdd} implies the original statement, note that given a continuous function $f$, one can construct a continuous function $g(x) = f(x) - f(-x)$ that is, by definition, odd. To see the opposite implication, we apply the original statement to the continuous odd function $g$ to find an $x \in \SS^n$ such that $g(x)=g(-x)$. Then, since $g$ is odd, $g(-x) = -g(x)$, which implies that $g(x)=0$.

There is a third classically equivalent statement.
\begin{statement}\label{BURetract}
There is no continuous odd function $h:\SS^n \rightarrow \SS^{n-1}$.
\end{statement}

In order to avoid issues involving the law of excluded middle, we will not prove this equivalence in homotopy type theory. Instead, our strategy for proving the Borsuk-Ulam Theorem will be to prove that Statement \ref{BURetract} is true and that it implies Statement \ref{BUOdd}, which is equivalent to Statement \ref{BUClassic}. 

We now state a classical/standard proof of Statement \ref{BURetract}, which inspired the type theoretic proof we give in this paper. 

\Amelia{I wasn't sure whether to put this in a proof environment or not since it is not a HoTT proof. We can decide this later.}

\Amelia{the proof doesn't cover n=1, but I think our intention here is just to give intuition for the proof strategy later, so thats ok. If we think it would be helpful to the reader later, we can add the n=1 case.}


 Let $h:\SS^n \rightarrow \SS^{n-1}$ be a continuous odd function with integer $n>2$ (\Amelia{does this work for n=2?}). Since $h$ is odd, we can identify antipodal points and arrive at the induced continuous function $h':\RRP^n\rightarrow\RRP^{n-1}$. Note that the fundamental groups of $\RRP^n$ and $\RRP^{n-1}$ are both $\ZZtwo$. 
 Thus, if the homomorphism of fundamental groups induced by $h'$ is nontrivial, it will also be an isomorphism.
 Since the fundamental group of $\RRP^n$ is $\ZZtwo$, it has a nontrivial element. Therefore there must exist a representative loop $\gamma$ in $\RRP^n$ that is essential. The loop $\gamma$ can then be lifted to $\SS^n$, resulting in a path $p$ between antipodal points in $\SS^n$ (since $\SS^n$ is the universal cover of $\RRP^n$). Applying the odd map $h$ to this path $p$, we get a path $h(p)$ between two antipodal points in $\SS^{n-1}$, which then projects to a loop $h'(\gamma)$ in $\RRP^{n-1}$. If $h'(\gamma)$ was null homotopic, the null homotopy would lift to $\SS^{n-1}$, contradicting that $h(p)$ is a path between antipodal points in $\SS^{n-1}$. 
 
 
 
  Using the Hurewicz homomorphism and universal coefficient theorem, we get an induced graded ring homomorphism on cohomology with $\ZZtwo$ coefficients, $h^*:H^*(\RRP^n;\ZZtwo)\leftarrow H^*(\RRP^{n-1};\ZZtwo)$, where $H^*(\RRP^n;\ZZtwo)\cong \ZZtwo[a]/\langle a^{n+1}\rangle$ and $H^*(\RRP^{n-1};\ZZtwo)\cong \ZZtwo[b]/\langle b^n\rangle$ (\Amelia{Why was this again?}).  Since $h':\RRP^n\rightarrow\RRP^{n-1}$ induces an isomorphism on the fundamental groups, and the Hurewicz homomorphism is an isomorphism at the $i=1$ level, the induced homomorphism $h^*$ must send $b$ to $a$. This implies that $b^n$ is sent to $a^n$, which is a contradiction since $b^n=0$ and $a^n\neq 0$.  Thus, no such map continuous odd map $h:\SS^n \rightarrow \SS^{n-1}$ can exist.  
 



\subsection{ Translating the classical Borsuk-Ulam Theorem into HoTT}


% =================================================
% SECTION: DEFINING RPn
% =================================================
\section{Topological and homotopical real projective spaces}
\label{sec:rpn}

OUTLINE:
\begin{itemize}
\item
  Define n-disks as both sets and types, the latter which is
  simply 1, since they're contractible. Show that $ \shape
  \DD = \D $ 
\item
  Define n-spheres as sets.  Use pushouts to glue
  disks together. Explain why we need to glue with a
  collar---i.e. the ``topology'' (as encoded by continuous
  paths $ \RR \to \X $ of a type $ \X $. Show, via Shulman,
  that $ \shape \SS^n = \S^n $ 
\item
  Define $ \RRP^n $ as sets using pushouts and collaring.
  Recall Bulcholtz and Egbert's definition of HIT $ \RP^n
  $. \Chandrika{First draft done}
 \item
  Prove that $ \shape \set{RP}^n = \RP^n $, using an idea like "shape preserves pushouts".
\item
Show that the homology of $\set{RP}^n$ is what we want it to be (see Prop 5.1, in Amelia's section).
\end{itemize}

\subsection{Disks, Spheres and their shapes}
\subsection{Defining $\set{RP}^n$}
We define $\set{RP}^n$ using push outs, tautological bundles, spheres, and an inductive process, similar to the work of Rijike [].

The base case for our induction is $\set{RP}^1$.

\begin{definition}
We define $\set{RP}^1$ as the push out seen in the diagram below, where $\set{I}$ is the open unit interval, and the maps are defined by $a_1(-1,x) = \frac{x}{4}$ and $a_1(1,x)= \frac{x}{4} + \frac{3}{4}$, and $b_1 (\pm 1, x) := (0, a(\pm1, x))$.\\
\[\begin{tikzcd}
\set{S}^0 \times \set{I} \arrow[r, hook, "a_1"] \arrow[d, hook, "b_1" ] & \set{I} \arrow[d]\\
\{0 \} \times \set{I} \arrow[r] & \set{RP}^1
\end{tikzcd}
\]
\end{definition}

It is helpful to think of $a_1$ as an inclusion of a tubular neighbourhood of the boundary of $[0,1] \subseteq \set{R}$, with the boundary points deleted (so that it is indeed a subset of the open interval $\set{I}$). We call this a \emph{thickened boundary}. In fact, $b_1$ is also an inclusion of a thickened boundary. The image of $b_1$ is the deleted tubular neighbourhood of the boundary of $\{0\} \times [0,1] \subseteq \{0\} \times \set{R}$.

\begin{definition}
Let $A, B$ be two cohesive types, with $A$ an open subset of $B$. Let $\bar{A}$ be the closure of $A$ in $B$. Suppose $\bar{A}$ has a non-empty boundary, $\partial \bar{A}$. Let $N'$ be a tubular neighbourhood of $\partial\bar{A}$. Let $N := N' \setminus A^C$, where $A^C$ is the complement of $A$. Even though $A$ has no boundary, we call $N$ a \emph{thickened boundary of $A$, relative to $B$}.
\end{definition}
%%% show that shape of \set{RP1} is S1.. start reading pg 63 of brouwer fixed pt paper.%%

We also need the following $\set{R}$-bundle as part of our base case. 
\begin{definition}(M\"{o}bius strip)
Let $\U_{\set{R}}$ be the universe of types homeomorphic to $\set{R}$. The \emph{tautological bundle} $\text{taut}^1: \set{RP}^1 \rightarrow \U_{\set{R}}$ is defined by the constructors $\set{I} \mapsto \set{R}$, $\{0 \} \times \set{I} \mapsto \set{R}$, and the equivalence $\lambda x. x$ for every point in $\{0\} \times \set{I}$, and the equivalence $\lambda x. -x$ for every point in $\{1\} \times \set{I}$.
\end{definition}

We note that the homeomorphism $\tau: \set{R} \rightarrow \set{I}$, $\tau x. \text{arctan}(x)+ \frac{1}{2}$, can be used to define a thickened boundary of $\text{taut}^1$. We define $\tau(\text{taut}^1)$ as a $\set{I}$-bundle given by the composition $\tau \circ \text{taut}^1$.  Note that there is a natural inclusion $\tau(\text{taut}^1) \hookrightarrow \text{taut}^1$ induced by the inclusion of $\set{I}$ into $\set{R}$. Any thickened boundary of $\tau(\text{taut}^1)$ relative to $\text{taut}^1$ has a pre-image by $\tau$. With a slight abuse of notation, we call this a thickened boundary of $\text{taut}^1$. \Chandrika{Is this correct language? (Where $\text{taut}^1$ is used as the map defining a bundle and also as the bundle itself?)}

From here, we proceed inductively with the definitions below. The definition of $\set{RP}^n$ depends on the definition of $\text{taut}^{n-1}$, which depends on $\set{RP}^{n-1}$, which depends on $\text{taut}^{n-2}$, and so on.

\begin{definition} $\set{RP}^n$ is defined (up to homeomorphism) as the following pushout.\\
\[\begin{tikzcd}
\set{S}^{n-1} \times \set{I} \arrow[r, hook, "a_n"] \arrow[d, hook, "b_n" ] &\set{D}^{n} \arrow[d]\\
\text{taut}^{n-1} \arrow[r] &\set{RP}^n\\
\end{tikzcd}
\] where $a_n$ and $b_n$ are both inclusions of $\set{S}^{n-1} \times \set{I} $ as thickened boundaries of $\text{taut}^{n-1}$ and $\set{D}^{n}$ (relative to $\set{R}^n$). The choice of thickened boundary does not change the homeomorphism type of $\set{RP}^n$.
\end{definition}
\Chandrika{Are others worried about only defining it up to homeomorphism? My thinking was that in type theory, things are only really ever defined up to equivalence, because of univalence.}

\begin{definition}
The tautological bundle $\text{taut}^n: \set{RP}^n \rightarrow \U_{\set{R}}$ is defined by the constructors $\set{D}^n \mapsto \set{R}$ and $\text{taut}^{n-1} \mapsto \set{R}$, and the automorphism of $\set{R}$, $\lambda x.x$ for every point of $\set{S}^{n-1} \times \set{I}$.
\end{definition}

%%%


\subsection{Showing $ \shape \set{RP}^n = \RP^n $}
Our first observation is that shape is a left adjoint operator, and so it preserves colimits. Therefore we have the following commutative diagram.
\[\begin{tikzcd}
\shape(\set{S}^{n-1} \times \set{I}) \arrow[r] \arrow[d] &\shape (\set{D}^{n}) \arrow[d]\\
\shape (\text{taut}^{n-1}) \arrow[r] &\shape (\set{RP}^n)\\
\end{tikzcd}
\] 

Based on work of Shulman \cite{}, we have $\shape(\set{S}^{n-1} \times \set{I}) =S^{n-1}$ and $\shape (\set{D}^{n})$ is the unit type. Next, we find $\shape (\text{taut}^{n-1})$ so that we may understand $ \shape \set{RP}^n$.  We begin by showing that $\text{taut}^{n}$ is a push out for all $n$. Consider the following commutative diagram. 


\subsection{The $\mathbb{Z}$-homology of $\set{RP}^n$.}

% =================================================
% SECTION: COHOMOLGY
% =================================================
\section{Cohomology}
\label{sec:cohom}

OUTLINE:
\begin{itemize}
\item 
  \textbf{Subsection 1.} 
  Define cohomology for $ \ZZtwo $ coefficients and the
  EM-spaces for $ \RRP^n $ \Amelia{Done except for some questions/references/anxiety and maybe a proof of prop 5.4}
\item
  \textbf{Subsection 2.} Show that we get a commutative
  graded ring structure for cohomology of any type $ \X $
  with $ \ZZtwo $-coefficients. Follow Brunerie's thesis.
\item
  \textbf{Subsection 3.} Compute $ \ZZtwo $-cohomology ring
  for $ \RRP^n $ using Mayer-Vietoris.  To do this we 
  first need to compute cohomology for disks and spheres.  
\end{itemize}


\subsection{Cohomology and EM-spaces for $\RRP^n$}


We follow a similar construction for cohomology as found in \cite{brunerie:thesis}, modifying their construction with $\ZZ$ coefficients to have coefficients in $\ZZtwo$. In order to define cohomology, we must first define Eilenberg-MacLane spaces $\KZtwo{n}$. Eilenberg-MacLane spaces $K(G,n)$ were defined for arbitrary group $G$ by Finster and Licata in \cite{fl:em}.

We give a construction of $\RRP^n$ in Section \ref{sec:rpn}, which shows that $\RRP^2$ will have the homotopy groups necessary to be the foundation of our Eilenberg-MacLane spaces. 
\begin{proposition}\label{prop:rpnhtpygps} %update this label when sec:rpn is done
	\[ \pi_n(\RRP^2)=\left\{ \begin{array}{ll} 0 & \text{for }n=0\\
	\ZZtwo & \text{for } n=1\\
	0 & \text{for } n> 1, 
	\end{array}\right.\]
\end{proposition}
\begin{proof}
	\Amelia{We discussed this on Oct 1 2019 and thought this would go in Chandrika's section}
\end{proof}


\begin{definition}\label{def:EM-space}
	For $n:\NN$, the type \defn{Eilenberg-MacLane space} $\KZtwo{n}$ is the $n$-truncated and $(n-1)$-connected pointed type defined by
	\[ \KZtwo{n}:=\left\{ \begin{array}{ll} \ZZtwo & \text{for }n=0\\
	\trunc{\Sigma^{n-1}\RRP^2}_n & \text{for } n\ge 1, 
	\end{array}\right.\]
	where $\Sigma^{n-1}$ indicates the reduced suspension $n-1$ times. 
\end{definition}

\begin{proposition}
	This definition of $\KZtwo{n}$ does indeed define an Eilenberg-MacLane Space. 
\end{proposition}

\begin{proof} This follows from Prop \ref{prop:rpnhtpygps} and
	\Amelia{the fact (that I don't have a HoTT reference for) that EM spaces are unique.}
\end{proof}

\Amelia{Finster and Licata constructed $K(G,1)$ in a particular way. My concern is that I am not sure how that impacts which of their results transfer over. I think that their results in section 4 and 5 mean that as long as the $K(G,1)$ has the right groups, using suspension and truncation works. In which case, it doesn't matter that we constructed $K(G,1)$ differently than they did, their results should all port over. I'd really like a second opinion on this. }
\Amelia{On October 17 2019 we talked about this question of what results port over. we said: According to Mike, in classic topology, EM spaces are unique up to unique isomorphism, however in HoTT they are equivalent up to homotopy but the homotopy may not be unique. Our thought was that our K(G,1) should be homotopy equivalent to theirs, so by univalence they're equal as types. So, their results for their construction of K(G,1) should apply to ours.  }

As noted in \cite{fl:em}, Eilenberg-Maclane spaces have the following delooping property. This property allows us to construct a spectrum for ordinary cohomology with $\ZZtwo$ coefficients. 

\begin{proposition}\label{prop:EM-loop-equiv}
	\[K(\ZZtwo,n)\simeq \Omega K(\ZZtwo,n+1)\]
	\end{proposition}
\Amelia{ **** G showed equivalence of $K_n$ and $\Omega K_{n+1}$. I thought this was in Finster and Licata. But now I can only find a mention of it as a fact, not something proven. May need to go back and prove this. }

Given this construction of the EM spaces, we define cohomology in the following way. 

\begin{definition}
	For a type $\X$ and $n:N$, the \defn{$n$-th cohomology group of $\X$} is the type
	\[H^n(\X;\ZZtwo):= \trunc{\X\rightarrow \KZtwo{n}}_0.\] 
\end{definition}
\Amelia{B also defines reduced cohomology as well. I'm not sure we need that. G proved a reduced MV sequence, so we may need it if we also use MV. Leaving it out for now.}



\subsection{Commutative Graded Ring Structure}



 Our strategy for describing the graded ring structure is to show that $||K(R,n)\wedge K(R,m)||_{n+m} = K(R\otimes R, n+m)$, then  lift the ring multiplication of $R \otimes R \to R$ to a cup product on cohomology with $R$-coefficients.


\Amelia{ basically copy pasted the following from ``Notes'', with some edits. I'll keep working through this.}

\begin{proposition}\label{prop:K_smash}
	$ \trunc{K(R,n) \wedge K(R,m)}_{n+m} = K(R \otimes R, n+m)$
\end{proposition} 
\Amelia{is = the right symbol here? probably yes bc univalence?}

\begin{proof}
	Recall that $K(R \otimes R, n+m)$ is the unique up to homotopy space
	having the property that
	\[
	\pi_k (K(R \otimes R, n+m )) \cong
	\begin{cases}
	0, & k \neq n+m \\
	R \otimes R, & k= n+m \\
	\end{cases}
	.\]
	\Amelia{on 12/5, we decided to go with isomorphism symbols for groups instead of equalities. Univalence most likely means that we could write equals, but there doesn't seem to be a need.}
	
	Thus, by showing that $ || K(R,n) \wedge K(R,m) ||_{n+m} $
	satisfies this property, we have the desired
	equivalence. 
	
	In order to establish the property for $k=n+m$, it is a known result (see \cite[Prop 19.60]{strom:mcht}) that
	\[
	\pi_{n+m} ( || K(R,n) \wedge K(R,m) ||_{n+m} )
	\cong R \otimes R.
	\]
	
	In the case where $k>m+n$, 
	 $ || K(R,n) \wedge K(R,m) ||_{n+m} $
	is truncated to be an $ ( n+m ) $-type. So, \[
	\pi_k ( || K(R,n) \wedge K(R,m) ||_{n+m} )
	\cong 0,
	\] for $k>n+m$. 
	
	Finally, note that  
	$ || K(R,n) \wedge K(R,m) ||_{n+m} $ is
	$ ( n+m-1 ) $-connected, which follows from $ K( R,k ) $ being
	$ ( k-1 ) $-connected, as seen in \cite[Prop 4.3.1]{brunerie:thesis}.
	Thus,
	\[
	\pi_k (|| K(R,n) \wedge K(R,m) ||_{n+m} ) \cong 0
	\]
	for $ k \leq n+m-1 $.
	
	Thus we have shown the desired property and arrive at the homotopy equivalence 
	\[
	|| K(R,n) \wedge K(R,m) ||_{n+m} \simeq
	K (R \otimes R, n+m ).
	\]
	Invoking the univalence axiom, we conclude that
	\[
	|| K(R,n) \wedge K(R,m) ||_{n+m} =
	K (R \otimes R, n+m ).
	\]
\end{proof}



Next, we construct the ring structure on cohomology. The
strategy is to define the necessary structures and properties on EM-spaces (Def. \ref{def:EM-space})
and then lift those to cohomology.



The addition and subtraction
operations of type $ K(\ZZtwo,n) \times K(\ZZtwo,n) \to K(\ZZtwo,n) $ follow
directly from \cite{brunerie:thesis}, and satisfy the properties to make $K(\ZZtwo,n)$ an abelian group. We restate the operation below.


\begin{proposition}\cite[Prop. 5.1.4]{brunerie:thesis}
	The maps $ +: K(\ZZtwo,n) \times K(\ZZtwo,n) \to K(\ZZtwo,n) $ and $-: K(\ZZtwo,n) \times K(\ZZtwo,n) \to K(\ZZtwo,n) $ are given by the following equalities for all $x$, $y$, $z:K(\ZZtwo,n)$ :
	
	\[x+0=x,\]
	\[0+x=x,\]
	\[x+(-x)=0,\]
	\[(-x)+x=0,\]
	\[(x+y)+z=x+(x+z),\]
	and
	\[x+y=y+x.\]
	
	\end{proposition}



We now define the ring operation multiplication. From the multiplication
$ \ZZtwo \otimes \ZZtwo \to \ZZtwo $, we induce a multiplication
$ K(\ZZtwo \otimes \ZZtwo, n)\to K(\ZZtwo, n) $ by applying the
functor
\[
K(-,n) \from \mathbf{Ring} \to \mathbf{Ho}(\mathbf{Top_\ast}).
\]
\Amelia{Do we know that $K(-,n)$ is a functor? I think so, but I'm not sure we ever proved it and I don't think it's explicitly stated in Finster-Licata.}

As we showed in Proposition \ref{prop:K_smash}, 
\[
|| K(R,n) \wedge K(R,m) ||_{n+m} =
K (R \otimes R, n+m ).
\]
for any ring $ R $ , and so we can treat our multiplication as
\[
\widehat{\mu} \from
|| K(\ZZtwo,n) \wedge K(\ZZtwo,m) ||_{n+m} \to K(\ZZtwo, n+m).
\]


However, we want the domain of multiplication
to be $ K ( \ZZtwo , n) \times K( \ZZtwo, m) $, without the truncation. To get this, we
precompose $ \widehat{\mu} $ by several canonical arrows
resulting in the composite
\[
K (\ZZtwo , n) \times K(\ZZtwo, m)
\xrightarrow{\mathtt{proj}}
K (\ZZtwo , n) \wedge K(\ZZtwo, m)
\xrightarrow{|-|_{n+m}}\cdots\]
\[\rightarrow|| K (\ZZtwo , n) \wedge K(\ZZtwo, m) ||_{n+m}
\xrightarrow{\widehat{\mu}}
K(\ZZtwo, n+m)
\]
that we call \emph{multiplication} $ \mu $.  We will use this multiplication to induce
the cup product
\[
\smile \from H^n(X) \times H^m(X) \to H^{n+m}(X)
\]
on cohomology. 



Recall that we define the $ n $-th cohomology
with $ \ZZtwo $-coefficients to be
$H^n(\X;\ZZtwo)= \trunc{\X\rightarrow \KZtwo{n}}_0$. This is the
standard definition in homotopy type theory, rather than
singular cochains, because using singular cochains is not
a continuous process. 

\Amelia{I've worked this far as of 2/27/20}

Define
$ \smile ( |\alpha|, |\beta|) ) $, for
$ \alpha \from X \to_\ast K_n $ and
$ \beta \from X \to K_m $, to be the truncation of the
pairing $ \langle \alpha, \beta \rangle $ followed by $ \mu $:
\[
\smile ( |\alpha|, |\beta|) ) :
|| X \to K_n \times K_m \to K_{n+m} ||_0.
\]
Thus $ \smile ( |\alpha|, |\beta|) ) $ type-checks. It
remains to show the usual ring properties hold for $ H^n(X)
$.  Distributivity follows directly from Guillaume's
argument in Proposition 5.1.7 which uses only connectivity
hence carries through to our context without issue.  To show
`graded comutativity', it suffices to show standard
comutativity because of the $ \ZZtwo $-coefficients.  





\subsection{Computing the Cohomology Ring of $\RRP^n$}




% =================================================
% SECTION: PROVING THE BORSUK ULAM THEORM
% =================================================
\section{The Borsuk-Ulam theorem}
\label{sec:borsuk-ulam}

OUTLINE:
\begin{itemize}
\item
  The proof is done by this point. Just put it all
  together and reconnect the dots for the reader.
\end{itemize}


% =================================================
% BIBLIOGRAPHY
% =================================================
\bibliographystyle{alpha}
\nocite{shul:bfp,brunerie:thesis,br:rp-hott}
\bibliography{assets/appcoh}


\end{document}