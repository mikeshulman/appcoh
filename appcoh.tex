\documentclass{amsart}
% This is the preamble for the
% paper proving the Borsuk-Ulam
% theorem

% ~~~~~~~~~~~~~~~~~~~~~~~~~~~~~
% PACKAGES 
% ~~~~~~~~~~~~~~~~~~~~~~~~~~~~~
\usepackage{amsfonts}
\usepackage{amssymb}  
\usepackage{amsthm} 
\usepackage{amsmath} 
\usepackage{caption}
\usepackage[inline]{enumitem}
  \setlist{itemsep=0em, topsep=0em, parsep=0em}
  \setlist[enumerate]{label=(\alph*)}
\usepackage{etoolbox}
\usepackage{stmaryrd} 
\usepackage[dvipsnames]{xcolor}
  \definecolor{editcolour}{rgb}{0.7,0.1,0}
  \definecolor{hrefcolour}{rgb}{0,0,0.7}
  \definecolor{Amelia}{rgb}{1,0,0}
  \definecolor{Chandrika}{rgb}{0,1,0}
  \definecolor{Daniel}{rgb}{0,0,1}
\usepackage[]{hyperref}
  \hypersetup{%
    colorlinks,
    linkcolor={hrefcolour},
    citecolor={hrefcolour},
    urlcolor={hrefcolour}}
\usepackage{graphicx}
  \graphicspath{ {assets/} }
\usepackage{mathtools}
\usepackage{tikz}
\usetikzlibrary{%
  matrix,
  arrows,
  shapes,
  decorations.markings,
  decorations.pathreplacing}

% ~~~~~~~~~~~~~~~~~~~~~~~~~~~~~~~
% NEW COMMANDS
% ~~~~~~~~~~~~~~~~~~~~~~~~~~~~~~~

% text formatting
\newcommand{\defn}[1]{\textbf{#1}} % to define a word
\newcommand{\cat}[1]{\mathrm{#1}} % cateegory font
\newcommand{\type}[1]{\mathtt{#1}} % type font

% sets and types
\newcommand{\A}{\type{A}}
\renewcommand{\AA}{\mathbb{A}}
\newcommand{\B}{\type{B}}
\newcommand{\BB}{\mathbb{B}}
\newcommand{\C}{\type{C}}
\newcommand{\CC}{\mathbb{C}}
\newcommand{\D}{\type{D}}
\newcommand{\DD}{\mathbb{D}}
\newcommand{\E}{\type{E}}
\newcommand{\EE}{\mathbb{E}}
\newcommand{\F}{\type{F}}
\newcommand{\FF}{\mathbb{F}}
\newcommand{\G}{\type{G}}
\newcommand{\GG}{\mathbb{G}}
\renewcommand{\H}{\type{H}}
\newcommand{\HH}{\mathbb{H}}
\newcommand{\I}{\type{I}}
\newcommand{\II}{\mathbb{I}}
\newcommand{\J}{\type{J}}
\newcommand{\JJ}{\mathbb{J}}
\newcommand{\K}{\type{K}}
\newcommand{\KK}{\mathbb{K}}
\renewcommand{\L}{\type{L}}
\newcommand{\LL}{\mathbb{L}}
\newcommand{\M}{\type{M}}
\newcommand{\MM}{\mathbb{M}}
\newcommand{\N}{\type{N}}
\newcommand{\NN}{\mathbb{N}}
\renewcommand{\O}{\type{O}}
\newcommand{\OO}{\mathbb{O}}
\renewcommand{\P}{\type{P}}
\newcommand{\PP}{\mathbb{P}}
\newcommand{\Q}{\type{Q}}
\newcommand{\QQ}{\mathbb{Q}}
\newcommand{\R}{\type{R}}
\newcommand{\RR}{\mathbb{R}}
\renewcommand{\S}{\type{S}}
\renewcommand{\SS}{\mathbb{S}}
\newcommand{\T}{\type{T}}
\newcommand{\TT}{\mathbb{T}}
\newcommand{\U}{\type{U}}
\newcommand{\UU}{\mathbb{U}}
\newcommand{\V}{\type{V}}
\newcommand{\VV}{\mathbb{V}}
\newcommand{\W}{\type{W}}
\newcommand{\WW}{\mathbb{W}}
\newcommand{\X}{\type{X}}
\newcommand{\XX}{\mathbb{X}}
\newcommand{\Y}{\type{Y}}
\newcommand{\YY}{\mathbb{Y}}
\newcommand{\Z}{\type{Z}}
\newcommand{\ZZ}{\mathbb{Z}}

% other sets and types
\newcommand{\ZZtwo}{\ZZ/2\ZZ} % Z mod two
\newcommand{\RP}{\type{RP}} % type real projective
\newcommand{\RRP}{\mathbb{R}\textrm{P}} % set real projective

% symbols
\renewcommand{\equiv}{\simeq}
\newcommand{\bydef}{\coloneqq} % definitional equality
\newcommand{\tin}{\colon} % \in for types
\newcommand{\ltrunc}{\left| \left|}
\newcommand{\rtrunc}{\right| \right|}
\newcommand{\shape}{\int}

% arrows
\newcommand{\from}{\colon}
\newcommand{\rel}{\nrightarrow}
\newcommand{\To}{\Rightarrow}
\newcommand{\xto}[1]{\xrightarrow{#1}}
\newcommand{\xgets}[1]{\xleftarrow{#1}}

% operators
\DeclareMathOperator{\Hom}{Hom}
\DeclareMathOperator{\id}{id}
\DeclareMathOperator{\colim}{colim}

% commenting
\newcommand{\Amelia}{{\color{Amelia} AMELIA:}}
\newcommand{\Chandrika}{{\color{Chandrika} CHANDRIKA:}}
\newcommand{\Daniel}{{\color{Daniel} DANIEL:}}

% ~~~~~~~~~~~~~~~~~~~~~~~~~~~~~~~~~~~
% ENVIRONMENTS & COUNTERS
% ~~~~~~~~~~~~~~~~~~~~~~~~~~~~~~~~~~~
\newtheorem{theorem}{Theorem}[section]
\newtheorem{lemma}[theorem]{Lemma}
\newtheorem{proposition}[theorem]{Proposition}
\newtheorem{corollary}[theorem]{Corollary}

\theoremstyle{remark}
\newtheorem{remark}[theorem]{Remark}
\newtheorem{notation}[theorem]{Notation}

\theoremstyle{definition}
\newtheorem{example}[theorem]{Example} 
\newtheorem{definition}[theorem]{Definition}

\setcounter{tocdepth}{1} % Sets depth for table of contents. 


\begin{document}

% =================================================
% TITLE & AUTHORS
% =================================================
\title{%
        The Borsuk-Ulam theorm in real-cohesive homotopy
        type theory}   
\author{%
        Daniel Cicala \and Chandrika Sadanand \and Michael
        Shulman \and Amelia Tebbe}
\begin{abstract}
        Borsuk-Ulam!
\end{abstract}
\maketitle

% =================================================
% SECTION: COMMENTARY
% =================================================
\section*{Writing Notes}
\label{sec:writing_notes}

Writing assignments:
\begin{itemize}
\item Amelia---section 5
\item Chandrika---section 4
\item Daniel---sections 2 and 3
\end{itemize}

Formalizing the cohomology proofs will be determined later.


% =================================================
% SECTION: INTRODUCTION
% =================================================
\section{Introduction}
\label{sec:intro}

% =================================================
% SECTION: OVERVIEW OF REAL COHESIVE HOTT
% =================================================
\section{Overview of real-cohesive homotopy type theory}
\label{sec:rc-hott}

OUTLINE:
\begin{itemize}
\item
  HoTT as foundations
\item
  Intepreting AlgTop theorems in HoTT is obsructed by
  discontinuous functions
\item
  Relating continuous and discontinuous with flat and
  sharp, which are borrowed from cohesive topoi
\item
  Formalizing flat and sharp in HoTT + axioms needed,
  e.g. Rflat
\item
  Connecting sets used in AlgTop with HITs used in HoTT
  via shape
\end{itemize}


% =================================================
% SECTION: TRANLATING BORSUK ULAM TO HOTT
% =================================================
\section{Translating Borsuk-Ulam to homotopy type theory}
\label{sec:bu-to-hott}

OUTLINE:
\begin{itemize}
\item
  \textbf{Subsection 1.} Give statements for BU-classic,
  BU-odd, BU-retract) a la wikipedia. The proof strategy:
  show BU-retract implies BU-odd which is equivalent to
  BU-classic, then prove BU-retract. Give the proof for
  BU-retract.
\item
  \textbf{Subsection 2.} Translate the classical statement
  into propositions as types. We want to model classical proof.
  The failure of contrpositive rule in constructive
  logic---(not q implies not p) is (p implies not not
  q)---means our proof strategy is BU-retract implies not
  not BU-odd which is equivalent to not not BU-classic. But
  not not BU-classic is sharp BU-classic. Prove BU-retract. 
\item
  To close out the section, list the ingredients we need to
  prove BU-retract.
\end{itemize}


% =================================================
% SECTION: DEFINING RPn
% =================================================
\section{Topological and homotopical real projective spaces}
\label{sec:rpn}

OUTLINE:
\begin{itemize}
\item
  Define n-disks as both sets and types, the latter which is
  simply 1, since they're contractible. Show that $ \shape
  \DD = \D $
\item
  Define n-spheres as sets.  Use pushouts to glue
  disks together. Explain why we need to glue with a
  collar---i.e. the ``topology'' (as encoded by continuous
  paths $ \RR \to \X $ of a type $ \X $. Show, via Shulman,
  that $ \shape \SS^n = \S^n $
\item
  Define $ \RRP^n $ as sets using pushouts and collaring.
  Recall Bulcholtz and Egbert's definition of HIT $ \RP^n
  $. Prove that $ \shape RRP^n = \RP^n $
\end{itemize}


% =================================================
% SECTION: COHOMOLGY
% =================================================
\section{Cohomology}
\label{sec:cohom}

OUTLINE:
\begin{itemize}
\item 
  {Subsection 1.} 
  Define cohomology for $ \ZZtwo $ coefficients and the
  EM-spaces for $ \RRP^n $
\item
  \textbf{Subsection 2.} Show that we get a commutative
  graded ring structure for cohomology of any type $ \X $
  with $ \ZZtwo $-coefficients. Follow Brunerie's thesis.
\item
  \textbf{Subsection 3.} Compute $ \ZZtwo $-cohomology ring
  for $ \RRP^n $ using Mayer-Vietoris.  This needs us to
  first compute cohomology for disks and spheres.  
\end{itemize}


% =================================================
% SECTION: PROVING THE BORSUK ULAM THEORM
% =================================================
\section{The Borsuk-Ulam theorem}
\label{sec:borsuk-ulam}

OUTLINE:
\begin{itemize}
\item
  The proof is done by this point. Just put it all
  together and reconnect the dots for the reader.
\end{itemize}


% =================================================
% BIBLIOGRAPHY
% =================================================
\bibliographystyle{alpha}
\nocite{shulman:bfp,brunerie:thesis,br:rp-hott}
\bibliography{assets/appcoh.bib}

\end{document}